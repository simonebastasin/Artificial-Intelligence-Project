\def\year{2022}\relax
%File: formatting-instructions-latex-2021.tex
%release 2021.2
\documentclass[letterpaper]{article} % DO NOT CHANGE THIS
\usepackage{aaai21}  % DO NOT CHANGE THIS
\usepackage{times}  % DO NOT CHANGE THIS
\usepackage{helvet} % DO NOT CHANGE THIS
\usepackage{courier}  % DO NOT CHANGE THIS
\usepackage[hyphens]{url}  % DO NOT CHANGE THIS
\usepackage{graphicx} % DO NOT CHANGE THIS
\urlstyle{rm} % DO NOT CHANGE THIS
\def\UrlFont{\rm}  % DO NOT CHANGE THIS
\usepackage{natbib}  % DO NOT CHANGE THIS AND DO NOT ADD ANY OPTIONS TO IT
\usepackage{caption} % DO NOT CHANGE THIS AND DO NOT ADD ANY OPTIONS TO IT
\usepackage{amsfonts}
\usepackage{amsmath}
\usepackage{amssymb}
\frenchspacing  % DO NOT CHANGE THIS
\setlength{\pdfpagewidth}{8.5in}  % DO NOT CHANGE THIS
\setlength{\pdfpageheight}{11in}  % DO NOT CHANGE THIS
%\nocopyright
%PDF Info Is REQUIRED.
% For /Author, add all authors within the parentheses, separated by commas. No accents or commands.
% For /Title, add Title in Mixed Case. No accents or commands. Retain the parentheses.
\pdfinfo{
    /Title (Matching models for evacuation and allocation of people in case of disasters and wars)
    /Author (Simone Bastasin, Simone Bortolin, Thomas Trevisan)
    /TemplateVersion (2021.2)
} %Leave this
% /Title ()
% Put your actual complete title (no codes, scripts, shortcuts, or LaTeX commands) within the parentheses in mixed case
% Leave the space between \Title and the beginning parenthesis alone
% /Author ()
% Put your actual complete list of authors (no codes, scripts, shortcuts, or LaTeX commands) within the parentheses in mixed case.
% Each author should be only by a comma. If the name contains accents, remove them. If there are any LaTeX commands,
% remove them.

% DISALLOWED PACKAGES
% \usepackage{authblk} -- This package is specifically forbidden
% \usepackage{balance} -- This package is specifically forbidden
% \usepackage{color (if used in text)
% \usepackage{CJK} -- This package is specifically forbidden
% \usepackage{float} -- This package is specifically forbidden
% \usepackage{flushend} -- This package is specifically forbidden
% \usepackage{fontenc} -- This package is specifically forbidden
% \usepackage{fullpage} -- This package is specifically forbidden
% \usepackage{geometry} -- This package is specifically forbidden
% \usepackage{grffile} -- This package is specifically forbidden
% \usepackage{hyperref} -- This package is specifically forbidden
% \usepackage{navigator} -- This package is specifically forbidden
% (or any other package that embeds links such as navigator or hyperref)
% \indentfirst} -- This package is specifically forbidden
% \layout} -- This package is specifically forbidden
% \multicol} -- This package is specifically forbidden
% \nameref} -- This package is specifically forbidden
% \usepackage{savetrees} -- This package is specifically forbidden
% \usepackage{setspace} -- This package is specifically forbidden
% \usepackage{stfloats} -- This package is specifically forbidden
% \usepackage{tabu} -- This package is specifically forbidden
% \usepackage{titlesec} -- This package is specifically forbidden
% \usepackage{tocbibind} -- This package is specifically forbidden
% \usepackage{ulem} -- This package is specifically forbidden
% \usepackage{wrapfig} -- This package is specifically forbidden
% DISALLOWED COMMANDS
% \nocopyright -- Your paper will not be published if you use this command
% \addtolength -- This command may not be used
% \balance -- This command may not be used
% \baselinestretch -- Your paper will not be published if you use this command
% \clearpage -- No page breaks of any kind may be used for the final version of your paper
% \columnsep -- This command may not be used
% \newpage -- No page breaks of any kind may be used for the final version of your paper
% \pagebreak -- No page breaks of any kind may be used for the final version of your paperr
% \pagestyle -- This command may not be used
% \tiny -- This is not an acceptable font size.
% \vspace{- -- No negative value may be used in proximity of a caption, figure, table, section, subsection, subsubsection, or reference
% \vskip{- -- No negative value may be used to alter spacing above or below a caption, figure, table, section, subsection, subsubsection, or reference

\setcounter{secnumdepth}{0} %May be changed to 1 or 2 if section numbers are desired.

% The file aaai21.sty is the style file for AAAI Press
% proceedings, working notes, and technical reports.
%

% Title

% Your title must be in mixed case, not sentence case.
% That means all verbs (including short verbs like be, is, using,and go),
% nouns, adverbs, adjectives should be capitalized, including both words in hyphenated terms, while
% articles, conjunctions, and prepositions are lower case unless they
% directly follow a colon or long dash

\title{Matching models for evacuation and allocation of people\\in case of disasters and wars}
\author{
%Authors
% All authors must be in the same font size and format.
    Written by Simone Bastasin, Simone Bortolin, Thomas Trevisan\textsuperscript{\rm 1}
    \\
}
\affiliations{
    \textsuperscript{\rm 1}Dipartimento di Ingegneria dell'informazione - Unipd \\
    Via Gradenigo 6, Padova, 35129, Italy \\

    simone.bastasin@studenti.unipd.it \\
    simone.bortolin.1@studenti.unipt.it \\
    thomas.trevisan@studenti.unipd.it \\
}

\begin{document}

    \maketitle

    \begin{abstract}
        In this article we analyze an important case study in machine learning: matching models for evacuation and
        allocation of people in case of disasters and wars. This case study was born mainly after two facts:
        the Swiss State Secretariat for Migration recently announced a pilot project for a machine
        learning-based assignment process for refugee resettlement~\citet{olbergml}.
        And the increasing number of terroristic attacks have forced the main emergency lighting companies to produce
        dynamic lamps that indicate which is the real emergency exit and not only the nearest one.

        In the last period moreover due to the increase of natural catastrophes and to the war in Ukraine it becomes
        more and more necessary to have matching models to manage in automatic way all these themes, that to the first
        sight could be very untied between them, but in reality they are extremely tied and similar.
    \end{abstract}

    \noindent%
    ~\citet{olbergml,basshuysen,delacretaz_2020,andersson}

    \subsection{Matching model}\label{matching-model}%
    ~\citet{olbergml,basshuysen,delacretaz_2020}

    \subsubsection{Matching model for Asylum}\label{matching-model-for-asylum}%
    ~\citet{olbergml,basshuysen,delacretaz_2020,fernandez} propose implementing a College Admissions (CA)
    model on the asylum market.
    The model differs from the EU model~\citet{basshuysen} because it looks not only at the preferences of the states
    that take people but also at the preferences of the refugees.
    It can be said that it is a generalized EU model.

    According to the notations of~\citet{salles} formally, a CA-instance of a refugee-country matching problem is a
    four-tuple \((C, R, q, P)\) (or if treated in a more realistic way, a 5-tuple \((C, R, q, P, F)\)), where
    \(C = \{c_1, \dots, c_m\}\) and \(R = \{r_1, \dots,r_n\}\) are disjoint sets of \(m\) countries and \(n\) refugees,
    respectively.
    The agents of the market, definition that comes from game theory field, are \(a_k \in C\cup R\),
    since it can be assumed that \(n \gg m\) it is also evident that it is a many-to-one matchings with an upper limit
    on the many side.
    The maximum number of refugees that can be matched to
    each country is determined by a vector of quotas \(q = (q_j)_{j\in {1,...,m}} \in \mathbb{N}^m\).
    You can also dummy the quotas in this way:  \(q_j = n \quad  \forall c_j \in C\).

    Afterwards we have, \(P = \{P(c_1), \dots , P(c_m), P(r_1), \dots , P(r_n)\}\) is
    a set of preference lists which induces a complete, transitive, and irreflexive preference profile for each
    country over the set of refugees and for each refugee over the set of countries.
    Each preference \(P(r_i)\) contains a list of expressed preferences in the format
    \( c _ { 1 } \succ_{r _ { i }} c _ { 2 } \),
    and equivalently for countries' preferences (an example is illustrated in Table~\ref{tab:countries-refugees}).

    In the particular case of refugees' allocation, there is one last element to consider: a list of people who must be
    together (e.g. a family) in this case you should be careful to remove duplicate entries of the same people.
    Also in this case there is a maximum limit to the number of people to keep in the same group.
    We then introduce \(F=\{F(f_1), \dots, F(f_l)\}\) where \(l\leq n\) and \(F(f_i) = \{r_a, r_b, \dots\}\).
    In the following, for simplicity, we will consider \(l=n\) and \(F(f_i)=\{r_i\}\).

    Countries may declare refugees unacceptable, and refugees may declare countries unacceptable,
    hence, there is a subset \(E \subseteq R \times C \times F\) of acceptable refugee-country pairs.
    In addition, there is the limitation on the amount of each individual tuple that is acceptable, as a refugee cannot
    be split into two nations nor can one nation receive more than the maximum allowable quotas.

    Denote \( A \left( r _ { i } \right) = \left\{ c _ { j } \mid \left( r _ { i } , c _ { j } \right) \in E \right\} \)
    the set of acceptable countries for a given \( r _ { i } \in R \); and equivalently for the countries.
    An assignment \(M\) is a subset of \(E\) and contains the item \( a _ { k } \in R \cup C \).
    Obviously a refugee \( r _ { i } \) can be unassigned so \( M \left( r _ { i } \right) = \emptyset \).
    Similarly, a country \( c _ { j } \) is
    under subscribed if \( \left| M \left( c _ { j } \right) \right| < q _ { j } \), and full if
    \( \left| M \left( c _ { j } \right) \right| = q _ { j } \).
    It is to remember that the assignment is valid if and only if:
    \( \left| M \left( r _ { i } \right) \right| \leq 1 \) for all \( r _ { i } \in R ; \) and
    \( \left| M \left( c _ { j } \right) \right| \leq q _ { j } \) for all \( c _ { j } \in C  \).


    \begin{table}[!htb]
        \begin{tabular}{c|c}
            \hline Countries                                          & Refugees                                                   \\
            \hline\( r_{1} \succ_{c_{1}} r_{2} \succ_{c_{1}} r_{3} \) & \( c_{2} \succ c_{1} \) for both \( r_{1} \) and \( r_{2} \) \\
            \( r_{2} \succ_{c_{2}} r_{1} \succ_{c_{2}} r_{3} \)       & \( r_{3} \) declares only \( c_{2} \) acceptable           \\
            \hline
        \end{tabular}
        \caption{Table specifying refugees and countries preferences for
        the study case: there are three refugees, \( r _ { 1 } , r _ { 2 } , r _ { 3 } \), and two countries \( c _ { 1 } , c _ { 2 } \) with \( q _ { 1 } = 2 \) and \( q _ { 2 } = 1 \).
        Where \( a \succ_{c} b \) denotes that \( c \) strictly prefers \( a \) to \( b \).}
        \label{tab:countries-refugees}
    \end{table}

    \subsubsection{Matching model in case of disaster}\label{matching-model-in-case-of-disaster}%
    Almost every state has regulations for safety and construction of escape routes
    (\citet{it-81-2008,uk-1541-2005,usa-1910-1974,cee-654-1989,cee-567-1977}), in general these
    regulations can be summarized as: every 60\footnote{60 cm is the minimum size for a person to be
    able to walk without crawling or anything else} cm of corridor/door/staircase can come out around 50-70
    \footnote{In american regulations this number is always fixed at 60, in europe this number is more
    variable, usually it is 50, but it can go up to 70 in places of education or offices closed to the public}
    people, rounded to the nearest multiple of 60 cm.

    For example, a 120 cm wide escape route can allow the exit of 100--140 people, and a 90 cm
    escape route can allow the exit of 50--70 people.

    Another regulation is that from the center of a room to the nearest emergency exit or separate
    fire compartment there can be maximum 60 meters.
    These two combinations generate a matching model
    very similar and very correlated to that of Asylum.

    A CA-instance of a people-exit matching problem is a four-tuple \((E, P, q, N)\)
    (or if treated in a more realistic way, a 5-tuple \((E, P, q, N, F)\)), where
    \(E = \{e_1, \dots, e_m\}\) and \(P = \{p_1, \dots,p_n\}\) are disjoint sets of \(m\) fire exit and \(n\) people.
    The agents of the market are \(a_k \in P\subset E\), since it can be assumed that \(n \ll m\) it is also evident
    that it is a many-to-one matching with an upper limit  on the many side.
    The maximum number of people that can be matched to
    each fire exit is determined by a vector of quotas \(q = (q_j)_{j\in {1,...,m}} \in \mathbb{N}^m\).
    You can also dummy the quotas in this way:  \(q_j = n \forall c_j \in C\).

    Similarly to before we have, \(N = \{N(e_1), \dots , N(e_m), N(p_1), \dots , N(p_n)\}\) is
    a set of the nearest exit lists.
    Each nearest exit list \(N(p_i)\) contains a list of expressed preferences in the format
    \( e _ { 1 } \succ_{p _ { i }} e _ { 2 } \), and equivalently for exits' preferences (an example is illustrated in
    Table~\ref{tab:people-exit}).

    Also in the case of evacuation of people, there is one last element to consider a list of people who must be
    together (e.g. a family): \(F=\{F(f_1), \dots, F(f_l)\}\) where \(l\leq n\) and \(F(f_i) = \{r_a, r_b, \dots\}\).
    In the following, for simplicity, we will consider that \(l=n\) and that \(F(f_i)=\{r_i\}\).

    Fire exit may declare people unacceptable, and people may declare fire exit unacceptable,
    hence, there is a subset \(C \subseteq E \times P \times F\) of acceptable people-exit pairs.
    In addition, there is the limitation on the amount of each individual tuple that is acceptable, as a people
    cannot be split into two fire exit nor can one fire exit receive more than the maximum allowable quotas.

    Denote \( A \left( e _ { i } \right) = \left\{ p _ { j } \mid \left( e _ { i } , p _ { j } \right) \in C \right\} \)
    the set of acceptable people for a given \( e _ { i } \in E \); and equivalently for the people.
    An assignment \(M\) is a subset of \(C\) and contains the item \( a _ { k } \in E \cup P \).
    In the case of a disaster every person should be able to evacuate, so  \( p _ { i } \) can be unassigned and
    \( M \left( p _ { i } \right) \neq \emptyset \) although it is possible that for some reason that exit is impracticable.
    Just for this reason it is possible to realize models in which you have \(|M \left( e _ { i } \right)| \geq 1\)
    i.e. assign to each user a second emergency exit, so that if the first is impracticable the second can be used.
    This strategy is to be avoided, and it is better to have a system of detection of impassable exits.
    We omit the analysis of these models.
    Similarly, an exit \( e _ { j } \) is  under subscribed if
    \( \left| M \left( e _ { j } \right) \right| < q _ { j } \), and full if
    \( \left| M \left( e _ { j } \right) \right| = q _ { j } \).
    It is to remember that the assignment is valid if and only if:
    \( \left| M \left( p _ { i } \right) \right| = 1 \) for all \( p _ { i } \in P ; \) and
    \( \left| M \left( e _ { j } \right) \right| \leq q _ { j } \) for all \( e _ { j } \in E  \).

    \begin{table}[!htb]
        \begin{tabular}{c|c}
            \hline People                                             & Exit                                                       \\
            \hline\( p_{1} \succ_{e_{1}} p_{2} \succ_{e_{1}} p_{3} \) & \( e_{2} \succ e_{1} \) for both \( p_{1} \) and \( p_{2} \) \\
            \( p_{2} \succ_{e_{2}} p_{1} \succ_{e_{2}} p_{3} \)       & \( p_{3} \) declares only \( e_{2} \) within 60 mt         \\
            \hline
        \end{tabular}
        \caption{Table specifying people and exit preferences for
            the study case: there are three people, \( p _ { 1 } , p _ { 2 } , p _ { 3 } \), and two fire exit
            \( e _ { 1 } , e _ { 2 } \) with \( q _ { 1 } = 2 \) and \( q _ { 2 } = 1 \).}
        \label{tab:people-exit}
    \end{table}

    \subsubsection{Use of matching models in case of war}\label{use-of-matching-model-in-case-of-war}%
    The previous models (in~\ref{matching-model-for-asylum},~\ref{matching-model-in-case-of-disaster}), as well guessed
    are very important in case of war: the first model is used just
    to redistribute in a fair way people in case of mass escapes, the second one to optimize the evacuation
    towards bunkers or other safe places in case of attacks and attempts to buildings.
    The second model can be widely used to manage the evacuation and moving to safe locations of buildings, hospitals,
    entire campuses, cities and counties.
    In~\citet{delacretaz_2020,delacretaz_2019,delacretaz_2016}   there are other example cases of use of this models.

    \subsubsection{Similarities and differences between mechanisms}\label{similarities-and-differences-between-mechanisms}

    As we have seen the two previous models are very similar, there are few differences and it is possible to use a
    common system for the analysis and to realize the matching for both cases.
    in fact in~\citet{delacretaz_2020} the model has been generalized to be adepted in more situations.

    \subsection{Mechanisms}

    \subsubsection{Matching models as College Admissions Problem}\label{matching-model-as-college-admissions-problem}~\citet{basshuysen}

    \subsubsection{Matching models as School Choice Problem}\label{matching-model-as-school-choice-problem}~\citet{basshuysen}

    \subsubsection{Machine learning-based Matching}\label{machine-learning-based-matching}~\citet{olbergml}

    \subsubsection{Compliance with Ethical Principles}\label{compliance-with-ethical-principles}~\citet{basshuysen}

    ethicality in the case of the use of algorithms such as machine learning and error learning

    the use of machine learning algorithms is ethically flawed in all cases where the salvation of people is associated
    with a Bayesian algorithm since it is a probability that decides the salvation or death of people.


    \subsection{Conclusion}

    we have seen two models very adaptable to endless emergency situations and in case of disasters, we have also seen
    the main models of resolution and we have also seen how they can help people in case of disasters and war

    \subsection{References}

    \bibliography{citations}


\end{document}
