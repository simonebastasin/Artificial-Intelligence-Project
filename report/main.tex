\def\year{2022}\relax
%File: formatting-instructions-latex-2021.tex
%release 2021.2
\documentclass[letterpaper]{article} % DO NOT CHANGE THIS

\usepackage{aaai21}  % DO NOT CHANGE THIS
\usepackage{times}  % DO NOT CHANGE THIS
\usepackage{helvet} % DO NOT CHANGE THIS
\usepackage{courier}  % DO NOT CHANGE THIS
\usepackage[hyphens]{url}  % DO NOT CHANGE THIS
\usepackage{graphicx} % DO NOT CHANGE THIS
\usepackage{algorithm2e}
\urlstyle{rm} % DO NOT CHANGE THIS
\def\UrlFont{\rm}  % DO NOT CHANGE THIS
\usepackage{natbib}  % DO NOT CHANGE THIS AND DO NOT ADD ANY OPTIONS TO IT
\usepackage{caption} % DO NOT CHANGE THIS AND DO NOT ADD ANY OPTIONS TO IT
\usepackage{amsfonts}
\usepackage{amsmath}
\usepackage{amssymb}
\usepackage{subfiles}
\frenchspacing  % DO NOT CHANGE THIS
\setlength{\pdfpagewidth}{8.5in}  % DO NOT CHANGE THIS
\setlength{\pdfpageheight}{11in}  % DO NOT CHANGE THIS
%\nocopyright
%PDF Info Is REQUIRED.
% For /Author, add all authors within the parentheses, separated by commas. No accents or commands.
% For /Title, add Title in Mixed Case. No accents or commands. Retain the parentheses.
\pdfinfo{
    /Title (Matching models for evacuation and allocation of people in case of disasters and wars)
    /Author (Simone Bastasin, Simone Bortolin, Thomas Trevisan)
    /TemplateVersion (2021.2)
} %Leave this
% /Title ()
% Put your actual complete title (no codes, scripts, shortcuts, or LaTeX commands) within the parentheses in mixed case
% Leave the space between \Title and the beginning parenthesis alone
% /Author ()
% Put your actual complete list of authors (no codes, scripts, shortcuts, or LaTeX commands) within the parentheses in mixed case.
% Each author should be only by a comma. If the name contains accents, remove them. If there are any LaTeX commands,
% remove them.

% DISALLOWED PACKAGES
% \usepackage{authblk} -- This package is specifically forbidden
% \usepackage{balance} -- This package is specifically forbidden
% \usepackage{color (if used in text)
% \usepackage{CJK} -- This package is specifically forbidden
% \usepackage{float} -- This package is specifically forbidden
% \usepackage{flushend} -- This package is specifically forbidden
% \usepackage{fontenc} -- This package is specifically forbidden
% \usepackage{fullpage} -- This package is specifically forbidden
% \usepackage{geometry} -- This package is specifically forbidden
% \usepackage{grffile} -- This package is specifically forbidden
% \usepackage{hyperref} -- This package is specifically forbidden
% \usepackage{navigator} -- This package is specifically forbidden
% (or any other package that embeds links such as navigator or hyperref)
% \indentfirst} -- This package is specifically forbidden
% \layout} -- This package is specifically forbidden
% \multicol} -- This package is specifically forbidden
% \nameref} -- This package is specifically forbidden
% \usepackage{savetrees} -- This package is specifically forbidden
% \usepackage{setspace} -- This package is specifically forbidden
% \usepackage{stfloats} -- This package is specifically forbidden
% \usepackage{tabu} -- This package is specifically forbidden
% \usepackage{titlesec} -- This package is specifically forbidden
% \usepackage{tocbibind} -- This package is specifically forbidden
% \usepackage{ulem} -- This package is specifically forbidden
% \usepackage{wrapfig} -- This package is specifically forbidden
% DISALLOWED COMMANDS
% \nocopyright -- Your paper will not be published if you use this command
% \addtolength -- This command may not be used
% \balance -- This command may not be used
% \baselinestretch -- Your paper will not be published if you use this command
% \clearpage -- No page breaks of any kind may be used for the final version of your paper
% \columnsep -- This command may not be used
% \newpage -- No page breaks of any kind may be used for the final version of your paper
% \pagebreak -- No page breaks of any kind may be used for the final version of your paperr
% \pagestyle -- This command may not be used
% \tiny -- This is not an acceptable font size.
% \vspace{- -- No negative value may be used in proximity of a caption, figure, table, section, subsection, subsubsection, or reference
% \vskip{- -- No negative value may be used to alter spacing above or below a caption, figure, table, section, subsection, subsubsection, or reference

\setcounter{secnumdepth}{0} %May be changed to 1 or 2 if section numbers are desired.

% The file aaai21.sty is the style file for AAAI Press
% proceedings, working notes, and technical reports.
%

% Title

% Your title must be in mixed case, not sentence case.
% That means all verbs (including short verbs like be, is, using,and go),
% nouns, adverbs, adjectives should be capitalized, including both words in hyphenated terms, while
% articles, conjunctions, and prepositions are lower case unless they
% directly follow a colon or long dash

    %\bibliographystyle{plain}

\title{Matching models for evacuation and allocation of people\\in case of disasters and wars}
\author{
%Authors
% All authors must be in the same font size and format.
    Written by Simone Bastasin\textsuperscript{\rm 1}, Simone Bortolin\textsuperscript{\rm 1}, Thomas Trevisan\textsuperscript{\rm 1}
    \\
}
\affiliations{
    \textsuperscript{\rm 1}Dipartimento di Ingegneria dell'Informazione - Unipd \\
    Via Gradenigo 6, Padova, 35129, Italy \\

    simone.bastasin@studenti.unipd.it \\
    simone.bortolin.1@studenti.unipt.it \\
    thomas.trevisan@studenti.unipd.it \\
}

\begin{document}

    \maketitle

    \begin{abstract}
        In this article we analyze an innovative case study in machine learning: matching models for evacuation and allocation of people in case of disasters and wars. This topic has been devised mainly because of two nowadays problems. First, as described by~\citet{olbergml}, the Swiss State Secretariat for Migration recently announced a pilot project for a machine learning-based assignment process for refugee resettlement. Second, the increasing number of terrorist attacks have forced main companies in the emergency lighting sector to produce dynamic lamps to indicate which is the most suitable emergency exit and not only the nearest one.

        Moreover, in the last period, due to the increase of natural catastrophes and to the war in Ukraine it becomes even more necessary to have matching models that manage automatically  these kinds of problems. At first sight they might seem unrelated to each other but in reality they are extremely tied and similar.
    \end{abstract}

    \subsection{Introduction}\label{introduction}
    ~\citet{olbergml,basshuysen,delacretaz_2020,andersson_2018,andersson_2016}
    % TODO scrivere una introduzione e citare quelli sopra

    \subsection{Matching model}\label{matching-model}%
    ~\citet{olbergml,basshuysen,delacretaz_2020}
    % TODO riassunto sul Matching model  e citare quelli sopra
    % in particolare parlare di
    %1) Weakly Envy-free
    %2) Envy-Free
    %3) Stable
    %4) Non-Wasteful
    % sono spiegati quì
    %http://t8el.com/wp-content/uploads/2019/12/DKT-MMRR-Dec2019.pdf
    % e per citarlo \citet{delacretaz_2019} senza () e \cite{delacretaz_2019} con ()
    % citare l'immagine con \ref{fig:matching}
Matching under preferences is a tool from cooperative game theory. It can be applied to two-sided
markets in which heterogeneous agents, or goods, of one side are to be distributed over agents or
goods of the other side of the market, and the satisfaction of agents’ preferences, or respect for agents’
priorities, matter. Gale and Shapley laid the theoretical foundations for the theory.
Matching markets with sizes arise in a variety of contexts and in its simplest model, agents have ordinal preferences over objects that are available in multiple identical units. For each object, agents
are ranked according to exogenous priorities. Agents can have a size of either one or two:
single-unit agents require one unit of an object and double-unit agents require two units of the
same object. The model can be extended in various ways to fit specific applications in the following ways:
\begin{itemize}
\item constraints may take place over several dimensions
\item agents may have preferences over both an object and a number of
units
\item agents may desire units of different objects
\end{itemize}
 An agent and an object form a blocking pair of a
given matching if the agent prefers the object to his own and at least the number of units he requires are either unassigned or assigned to agents with a lower priority.
Stability, initially introduced by Gale and Shapley in 1962, is a central concept in matching theory. A matching
is stable if it does not have any blocking pair. In two-sided matching markets, with
strategic agents on both sides, stability constitutes an essential equilibrium criterion, thus an unstable matching is not at equilibrium. A blocking pair
is unfair in the sense that the agent is not able to get an object even though he has a high enough priority. 
    \begin{figure}[!htb]
        \def\svgwidth{\columnwidth}
        \input{matching.pdf_tex}
        \caption{Logical relationships among our solution concepts}
        \label{fig:matching}
    \end{figure}

    \subsubsection{Matching model for Asylum}\label{matching-model-for-asylum}%
    ~\citet{olbergml,basshuysen,delacretaz_2020,fernandez} propose the implementation of a College Admissions (CA) model for the refugee resettlement matching problem.
    The model differs from the EU model of~\citet{basshuysen} because it doesn't try to optimize only the employment success of a refugee in a certain state but it also takes into account refugees' preferences allowing a trade off between family welfare and overall employment success.
    It can be considered as a generalized EU model.

    According also to the formally notation of~\citet{salles}, a CA-instance of a refugee-country matching problem is a 5-tuple \((C, R, q, P, F)\), where \(C = \{c_1, \dots, c_m\}\) and \(R = \{r_1, \dots, r_n\}\) are disjoint sets of \(m\) countries and \(n\) refugees, respectively.

    Since the distribution of asylum is modelled as a matching problem under preferences, refugees and countries offering asylum make up a two-sided market in which the members of one side are to be distributed over members of the other side.
    So, we can define agents of the market as \(a_k \in C \cup R\).
    
    Note that we are concerned with \textit{many-to-one matchings} since it can be assumed that \(n \gg m\) and each refugee can obtain asylum in at most one country, whereas a given country can accept many refugees. The maximum number of refugees that can be matched to each country is determined by a vector of quotas \(q = (q_j)_j \in \mathbb{N}^m \quad j\in\{1,\dots ,m\}\). There may be no real quotas at all: setting \(q_j = n \quad \forall c_j \in C\) makes them dummies.

    Afterwards, \(P = \{P(c_1), \dots, P(c_m), P(r_1), \dots, P(r_n)\}\) is a set of preference lists which induces a complete, transitive, and irreflexive preference profile for each country over the set of refugees and for each refugee over the set of countries.
    Each preference \(P(r_i)\) contains a list of expressed preferences in the format \(c_1 \succ_{r_i} c_2\), and equivalently for countries' preferences (an example is illustrated in Table~\ref{tab:countries-refugees}).

    In the particular case of refugees' allocation, there is one last element to consider: a list of people who must be together (e.g. a family) and to do so let \(F=\{F(f_1), \dots, F(f_l)\}\) be the list of the groups of refugees, where \(l \leq n\) and \(F(f_i) = \{r_a, r_b, \dots\}\).
    In this case you should be careful to do not exceed the maximum limit to the number of people to keep in the same group and be careful to remove duplicate entries of the same people.
    In the following, for simplicity, we will consider \(l=n\) and \(F(f_i)=\{r_i\}\).

    Countries may declare refugees unacceptable and refugees may declare countries unacceptable, hence, there is a subset \(E \subseteq R \times C \times F\) of acceptable refugee-country pairs.
    In addition, there is the limit on the amount of each individual tuple that is acceptable, because, as a refugee can't be split into two nations nor can one nation receive more than the maximum allowable quotas.
    
    Denote \(A \left( r_i \right) = \left\{ c_j \mid \left( r_i, c_j \right) \in E \right\}\) the set of acceptable countries for a given \(r_i \in R\), and equivalently for the countries.
    
    An assignment \(M\) is a subset of \(E\) and contains the item \(a_k \in R \cup C\).
    Obviously a refugee \(r_i\) can be unassigned so \(M \left( r_i \right) = \emptyset\).
    Similarly, a country \(c_j\) admits applications if \(\left| M \left( c_j \right) \right| < q_j\), and it's full if \(\left| M \left( c_j \right) \right| = q_j\).
    Note that the assignment is valid if and only if \(\left| M \left( r_i \right) \right| \leq 1 \forall r_i \in R\) and \(\left| M \left( c_j \right) \right| \leq q_j \forall c_j \in C\)

    \begin{table}[!htb]
        \begin{tabular}{c|c}
            \hline Countries & Refugees \\
            \hline \(r_1 \succ_{c_1} r_2 \succ_{c_1} r_3\) & \(c_2 \succ c_1\) for both \(r_1\) and \(r_2\) \\
            \(r_2 \succ_{c_2} r_1 \succ_{c_2} r_3\) & \(r_3\) declares only \(c_2\) acceptable \\
            \hline
        \end{tabular}
        \caption{Refugees and countries preferences example: refugees \(r_1, r_2, r_3\); countries \(c_1, c_2\) with \(q_1 = 2\) and \(q_2 = 1\). Notation \(a \succ_c b\) denotes that \(c\) strictly prefers \(a\) to \(b\).}
        \label{tab:countries-refugees}
    \end{table}

    \begin{figure}[!htb] % TODO simonebastasin controlla la caption
        \def\svgwidth{\columnwidth}
        \input{complete_matching.pdf_tex}
        \caption{Matching in a market with two-dimensional constraints. There are five families
            \(f_1, \dots, f_5\), two localities \(c_1, c_2\), and two dimensions, represented by the left and right columns.
            The sizes of the families are \((3, 0), (1, 0), (1, 1), (1, 2)\), and \((2, 1)\)}
        \label{fig:complete_matching}
    \end{figure}

    \subsubsection{Matching model in case of disaster}\label{matching-model-in-case-of-disaster}%
    Almost every state has regulations for safety on construction of escape routes (\citet{it-81-2008,uk-1541-2005,usa-1910-1974,cee-654-1989,cee-567-1977}), these regulations can usually be summarized as: from every 60 cm\footnote{60 cm is the minimum distance for a person to be able to walk without crawling or anything else} of corridor/staircase can come out around 50-70 people\footnote{Regard regulations, in America this number is always fixed at 60, in Europe is more variable, usually it's 50 but it can go up to 70 in education facilities or private offices}.
    For example, a 120 cm wide escape route can allow the exit of 100--140 people, and a 90 cm escape route can allow the exit of 50--70 people.
    %Therefore, it can be rounded to the most intuitive integer number: 60 cm.

    Another regulation is that from the middle of a room to the nearest emergency exit or to the nearest separate fire compartment there can be maximum 60 meters.
    
    These two combinations generate a matching model very similar and very correlated to the one for refugee resettlement.

    A CA-instance of a people-exit matching problem is a 5-tuple \((E, P, q, N, F)\), where \(E = \{e_1, \dots, e_m\}\) and \(P = \{p_1, \dots, p_n\}\) are disjoint sets of \(m\) fire exit and \(n\) people.
    % to revise ---
    The agents of the market are \(a_k \in P \subset E\), since it can be assumed that \(n \ll m\) it is also evident that it is a many-to-one matching with an upper limit on the many side.
    The maximum number of people that can be matched to each fire exit is determined by a vector of quotas \(q = (q_j)_j \in \mathbb{N}^m\), \(j\in {1,...,m}\).
    You can also dummy the quotas in this way:  \(q_j = n \forall c_j \in C\).
    % ---

    Similarly to the matching model for asylum, let \(N = \{N(e_1), \dots, N(e_m), N(p_1), \dots, N(p_n)\}\) be the set of the nearest exit lists.
    Each nearest exit list \(N(p_i)\) contains a list of expressed preferences in the format \(e_1 \succ_{p_i} e_2\), and equivalently for exits' preferences (an example is illustrated in Table~\ref{tab:people-exit}).

    Also in the case of people evacuation, another element to consider is the list of people who must be together (e.g. a family): \(F=\{F(f_1), \dots, F(f_l)\}\) where \(l\leq n\) and \(F(f_i) = \{r_a, r_b, \dots\}\).
    In the following, for simplicity, we will consider that \(l=n\) and that \(F(f_i)=\{r_i\}\).
    
    % simonebastasin ha modificato fino a qui

    Fire exit may declare people unacceptable, and people may declare fire exit unacceptable,
    hence, there is a subset \(C \subseteq E \times P \times F\) of acceptable people-exit pairs.
    In addition, there is the limitation on the amount of each individual tuple that is acceptable, as a people
    cannot be split into two fire exit nor can one fire exit receive more than the maximum allowable quotas.

    Denote \( A \left( e _ { i } \right) = \left\{ p _ { j } \mid \left( e _ { i } , p _ { j } \right) \in C \right\} \)
    the set of acceptable people for a given \( e _ { i } \in E \); and equivalently for the people.
    An assignment \(M\) is a subset of \(C\) and contains the item \( a _ { k } \in E \cup P \).
    In the case of a disaster every person should be able to evacuate, so  \( p _ { i } \) can be unassigned and
    \( M \left( p _ { i } \right) \neq \emptyset \) although it is possible that for some reason that exit is impracticable.
    Just for this reason it is possible to realize models in which you have \(|M \left( e _ { i } \right)| \geq 1\)
    i.e. assign to each user a second emergency exit, so that if the first is impracticable the second can be used.
    This strategy is to be avoided, and it is better to have a system of detection of impassable exits.
    We omit the analysis of these models.
    Similarly, an exit \( e _ { j } \) is  under subscribed if
    \( \left| M \left( e _ { j } \right) \right| < q _ { j } \), and full if
    \( \left| M \left( e _ { j } \right) \right| = q _ { j } \).
    It is to remember that the assignment is valid if and only if:
    \( \left| M \left( p _ { i } \right) \right| = 1 \) for all \( p _ { i } \in P ; \) and
    \( \left| M \left( e _ { j } \right) \right| \leq q _ { j } \) for all \( e _ { j } \in E  \).

    \begin{table}[!htb]
        \begin{tabular}{c|c}
            \hline People                                             & Exit                                                       \\
            \hline\( p_{1} \succ_{e_{1}} p_{2} \succ_{e_{1}} p_{3} \) & \( e_{2} \succ e_{1} \) for both \( p_{1} \) and \( p_{2} \) \\
            \( p_{2} \succ_{e_{2}} p_{1} \succ_{e_{2}} p_{3} \)       & \( p_{3} \) declares only \( e_{2} \) within 60 mt         \\
            \hline
        \end{tabular}
        \caption{Table specifying people and exit preferences for
            the study case: there are three people, \( p _ { 1 } , p _ { 2 } , p _ { 3 } \), and two fire exit
            \( e _ { 1 } , e _ { 2 } \) with \( q _ { 1 } = 2 \) and \( q _ { 2 } = 1 \).}
        \label{tab:people-exit}
    \end{table}

    \subsubsection{Use of matching models in case of war}\label{use-of-matching-model-in-case-of-war}%
    The previous models (in~\ref{matching-model-for-asylum},~\ref{matching-model-in-case-of-disaster}), as well guessed
    are very important in case of war: the first model is used just
    to redistribute in a fair way people in case of mass escapes, the second one to optimize the evacuation
    towards bunkers or other safe places in case of attacks and attempts to buildings.
    The second model can be widely used to manage the evacuation and moving to safe locations of buildings, hospitals,
    entire campuses, cities and counties.
    In~\citet{delacretaz_2020,delacretaz_2019,delacretaz_2016}   there are other example cases of use of this models.

    \subsubsection{Similarities and differences between mechanisms}\label{similarities-and-differences-between-mechanisms}

    As we have seen the two previous models are very similar, there are few differences and it is possible to use a
    common system for the analysis and to realize the matching for both cases.
    in fact in~\citet{delacretaz_2020} the model has been generalized to be adepted in more situations.

    \subsection{Mechanisms}

    \subsubsection{Matching models as College Admissions Problem}\label{matching-model-as-college-admissions-problem}~\citet{basshuysen}

    % da ~\citet{basshuysen} parlare del motivo per cui non va bene questo problema

    \subsubsection{Matching models as School Choice Problem}\label{matching-model-as-school-choice-problem}~\citet{basshuysen}

    % da ~\citet{basshuysen} parlare di quando è stato sviluppato il problema, di come si analizza

    % di cosa cambia tra la vecchia \(P\) ed le nuove
    \(\mathbf { P }= \{P(r_1), \dots, P(r_n)\}\)
    \(\mathbf { Pri } = \{Pri(c_1), \dots, Pro(c_m)\}\)
    5-tuple \( ( R , C , q , \mathbf { P } , \mathbf { Pri } ) \)
    % il resto della definizione è identico alla precedente

    \subsubsection{Machine learning-based Matching}\label{machine-learning-based-matching}~\citet{olbergml}

    % in ~\citet{bansak_2018} e \citet{olbergml} si parlano di due algoritmi

    To address this issue, we propose two mechanisms that build on the machine learning-based approach by~\citet{bansak_2018} the constrained random serial dictatorship mechanism
    (CRSD) and the constrained rank value mechanism (CRV)~\citet{olbergml} %continua in \citet{olbergml}

    \begin{algorithm}
        \caption{Constrained Random Serial Dictatorship (CRSD)}\label{alg:crsd}
        \KwData{\( F , L , \succeq , \pi , b , \alpha \)}
        \KwResult{\(\mu\)}
        \SetKw{Break}{break}
        \( \mu ( i ) \leftarrow \emptyset \) for all \( i \in F \).

        Compute the objective value \( z ^ { * } \) of an optimal solution to:
        \[\tiny
            \begin{aligned}
                \operatorname{maximize} & \sum_{i \in F} \sum_{j \in L} \pi_{i j} x_{i j} \\
                \operatorname {subject}\operatorname{to} & \sum_{i \in F} x_{i j}=q_{j} & \forall j \in L \\
                & \sum_{j \in L} x_{i j}=1 & x_{i \mu(i)}=1 \\
                & x_{i\mu (i)} = 1 &  \forall i \in F(\mu) \\
                & x_{i j} \in\{0,1\} & \forall i \in F, \forall j \in L
            \end{aligned}
        \]%

        \( Q \leftarrow F \).

        \While{\( Q \neq \emptyset \)}{
            Remove randomly chosen \( i \in Q \) from \( Q \).

            \If{\( \left| F _ { j } ( \mu ) \right| < q _ { j } \)}{
                \( \mu ^ { \prime } \leftarrow \mu ; \mu ^ { \prime } ( i ) \leftarrow j \)

                Solve
                \[\tiny
                    \begin{aligned}
                        \operatorname{maximize} & \sum_{i \in F} \sum_{j \in L} \pi_{i j} x_{i j} \\
                        \operatorname {subject}\operatorname{to} & \sum_{i \in F} x_{i j}=q_{j} & \forall j \in L \\
                        & \sum_{j \in L} x_{i j}=1 & x_{i \mu(i)}=1 \\
                        & x_{i\mu (i)} = 1 &  \forall i \in F(\mu) \\
                        & x_{i j} \in\{0,1\} & \forall i \in F, \forall j \in L
                    \end{aligned}
                \]%

                and let \( z ^ { \prime } \) denote the objective value of the solution.

                \If{\( z ^ { \prime } \geq \alpha z ^ { * } \)}{
                    \( \mu ( i ) \leftarrow j \)
                    \Break
                }
            }
        }
    \end{algorithm}

    \begin{algorithm}
        \caption{Constrained Rank Value Mechanism (CRV)}\label{alg:crv}
        \KwData{\( F , L , \succeq , \pi , b , \alpha \)}
        \KwResult{\(\mu\)}
        Compute the objective value \( z \left( \mu ^ { * } \right) \) of an optimal solution to
        \[\tiny
            \begin{aligned}
                \operatorname{maximize} & \sum_{i \in F} \sum_{j \in L} \pi_{i j} x_{i j} \\
                \operatorname {subject}\operatorname{to} & \sum_{i \in F} x_{i j}=q_{j} & \forall j \in L \\
                & \sum_{j \in L} x_{i j}=1 & x_{i \mu(i)}=1 \\
                & x_{i\mu (i)} = 1 &  \forall i \in F(\mu) \\
                & x_{i j} \in\{0,1\} & \forall i \in F, \forall j \in L
            \end{aligned}
        \]%
        Let \( \mu \) denote the matching induced by the solution of
        \[\tiny
            \begin{aligned}
                \operatorname{maximize} & \sum_{i \in F} \sum_{j \in L} v\left(\succeq_{i j}\right) x_{i j} \\
                \operatorname {subject}\operatorname{to} & \sum_{i \in F} x_{i j}=q_{j} & \forall j \in L \\
                & \sum_{j \in L} x_{i j}=1 & \forall i \in F \\
                & \sum_{i \in F} \sum_{j \in L} \pi_{i j} x_{i j} \geq \gamma \\
                & x_{i j} \in\{0,1\} & \forall i \in F, \forall j \in L
            \end{aligned}
        \]%
        with \( \gamma = \alpha z \left( \mu ^ { * } \right) \).
    \end{algorithm}

    \subsubsection{Compliance with Ethical Principles}\label{compliance-with-ethical-principles}~\citet{basshuysen}

    ethicality in the case of the use of algorithms such as machine learning and error learning

    the use of machine learning algorithms is ethically flawed in all cases where the salvation of people is associated
    with a Bayesian algorithm since it is a probability that decides the salvation or death of people.


    \subsection{Conclusion}

    we have seen two models very adaptable to endless emergency situations and in case of disasters, we have also seen
    the main models of resolution and we have also seen how they can help people in case of disasters and war

    \subsection{References}

    \bibliography{citations}


\end{document}
