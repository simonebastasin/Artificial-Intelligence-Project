\subsection{Use of matching models in case of war}\label{use-of-matching-model-in-case-of-war}%
Previous models argue topics that can be very useful in case of war. First model, for mass escapes, can redistribute in a quick and fair way people to safer countries.

Second model, during alarms for attacks, can optimally manage:
\begin{itemize}
    \item evacuation of people from buildings at risk of attack;
    \item shelter of people towards bunkers or other safe places.
\end{itemize}


\subsubsection{Similarities and differences between mechanisms}\label{similarities-and-differences-between-mechanisms}

As we have seen the two previous models are very similar, there are few differences so it is possible to use a
common system for the analysis and to realize the matching for both cases.
In fact ~\citet{delacretaz_2020} formulated a generalized model that can be adepted in more situations.
%in the table \ref{tab:summaryview} is a summary of the differences and similarities.
%
%\begin{table}[!hbt]
%    \centering
%    \begin{tabular}{p{0.15\columnwidth}|p{0.35\columnwidth}|p{0.35\columnwidth}}
%        \hline {} & {Model for asylum} & {Model for evacuation} \\
%        \hline Input matrix & \(C \): countries, \(R \): refugees & \(E \): exit, \( P \): people \\
%        \hline Input size & \(|C|= m\), \( |R|= n\), \( n \gg m \) & \(|E|= m\), \( |P|= n\), \( n \ll m \) \\
%        \hline Constrain & \(P_c\): preference of countries, \( P_r\): preference of refugee & \(N_e\): preference of exits, \( N_p\): preference of people  \\
%        \hline
%    \end{tabular}
%    \caption{Summary view of model similarities and differences.}
%    \label{tab:summaryview}
%\end{table}
