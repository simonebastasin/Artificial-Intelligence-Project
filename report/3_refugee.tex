\section{Matching models}\label{matching-models}%


\subsection{Matching model for refugee resettlement}\label{matching-model-for-refugee-resettlement}%

\citet{olbergml,basshuysen,delacretaz_2020,fernandez} propose the implementation of a model for the refugee resettlement matching problem.
The model differs from the EU model of \citet{basshuysen} because it doesn't try to optimize only the employment success of a refugee in a certain state but it also takes into account refugees' preferences allowing a trade off between family welfare and overall employment success.
It can be considered as a generalized EU model.

According also to the formally notation of \citet{salles}, a instance of a refugee-country matching problem is a 6-tuple \((C, R, q, P_c, P_r, F)\), where \(C = \{c_1, \dots, c_m\}\) and \(R = \{r_1, \dots, r_n\}\) are disjoint sets of \(m\) countries and \(n\) refugees, respectively.

Since the distribution for refugee resettlement is modelled as a matching problem under preferences, refugees and countries offering asylum make up a two-sided market in which the members of one side are to be distributed over members of the other side.
So, we can define agents of the market as \(a_k \in C \cup R\).

Note that we are concerned with \textit{many-to-one matchings} since it can be assumed that \(n \gg m\) and each refugee can obtain asylum in at most one country, whereas a given country can accept many refugees. The maximum number of refugees that can be matched to each country is determined by a vector of quotas \(q = (q_j)_j \in \mathbb{N}^m,\ j\in\{1,\dots ,m\}\). There may be no real quotas at all: setting \(q_j = n\ \ \forall c_j \in C\) makes them dummies.

Afterwards, \(P_c = \{P(c_1), \dots, P(c_m)\}\) and \(P_r =\{P(r_1), \dots, P(r_n)\}\) are sets of preference lists which include a complete and transitive preference profile for each country over the set of refugees and for each refugee over the set of countries.
Each preference \(P_r(r_i)\) contains a list of expressed preferences in the format \(c_1 \succ_{r_i} c_2\), and equivalently for countries' preferences (an example is illustrated in Table~\ref{tab:countries-refugees}).

In the particular case of refugees' allocation, the last element to consider is a list of people who must be together (e.g. a family): let \(F=\{F(f_1), \dots, F(f_l)\}\) be the list of the groups of refugees, where \(F(f_i) = \{r_a, r_b, \dots\}\) and \(l \leq n\).
In this case you should be careful to do not exceed the maximum bound to the number of people to keep in the same group and be careful to remove duplicate entries of the same people. An example of a full matching is shown in figure~\ref{fig:complete_matching}.

Countries may declare refugees unacceptable and refugees may declare countries unacceptable, hence, there is a subset \(E \subseteq R \times C \times F\) of acceptable refugee-country pairs.
In addition, the amount of an acceptable tuple is bounded because, as a refugee can't be split into two nations, nor can one nation receive more than the maximum allowable quotas.

Denote \(A \left( r_i \right) = \left\{ c_j \mid \left( r_i, c_j \right) \in E \right\}\) the set of acceptable countries for a given \(r_i \in R\), and equivalently for the countries.

An assignment \(M\) is a subset of \(E\) that contains \(a_k \in R \cup C\) items.
Obviously a refugee \(r_i\) can be unassigned: \(M \left( r_i \right) = \emptyset\).
Similarly, a country \(c_j\) admits asylum requests if \(\left| M \left( c_j \right) \right| < q_j\); it's full if \(\left| M \left( c_j \right) \right| = q_j\).
Note that the assignment is valid if and only if \(\left| M \left( r_i \right) \right| \leq 1\ \ \forall r_i \in R\) and \(\left| M \left( c_j \right) \right| \leq q_j\ \ \forall c_j \in C\)

\begin{table}[!htb]
    \centering
    \begin{tabular}{c|c}
        \hline Countries & Refugees \\
        \hline \(r_1 \succ_{c_1} r_2 \succ_{c_1} r_3\) & \(c_2 \succ c_1\) for both \(r_1\) and \(r_2\) \\
        \(r_2 \succ_{c_2} r_1 \succ_{c_2} r_3\) & \(r_3\) declares only \(c_2\) acceptable \\
        \hline
    \end{tabular}
    \caption{Refugees and countries preferences example. Refugees \(r_1, r_2, r_3\); countries \(c_1, c_2\) with \(q_1 = 2,\ q_2 = 1\). Notation \(a \succ_c b\) denotes that \(c\) strictly prefers \(a\) to \(b\).}
    \label{tab:countries-refugees}
\end{table}

\begin{figure}[!htb]
    \def\svgwidth{\columnwidth}
    \subfile{media/complete_matching.pdf_tex}
    \caption{Full matching example. Families \(f_1, \dots, f_5\) with \(|f_1|=|f_4|=|f_5|=3,\ |f_2|=1,\ |f_3|=2\); countries \(c_1, c_2\) with \(q_1 = 6,\ q_2 = 7\).}
    \label{fig:complete_matching}
\end{figure}
