\section{Mechanisms}\label{mechanisms}%


\subsection{Matching models as College Admissions Problem}\label{matching-model-as-college-admissions-problem}%

A CA-instance of a refugee-country matching problem is a 4-tuple \((C, R, q, P)\) where \(P = \{P_c, P_r\}\) and \(F\) is omitted.
Similarly, a CA-instance of a exit-people matching problem is a 4-tuple \((E, P, q, N)\) where \(N = \{N_e ,  N_p\}\) and \(F\) is omitted.

We can now turn to the desiderata encountered in the previous section.
It will be argued that the desideratum that preferences be satisfied “as much as possible" amounts to stability in the CA
model.
We shall first define this property and introduce an algorithm that produces a stable matching~\cite{basshuysen}.

The CA model meets the requirements of stability and maximum cardinality, but for the case study of asylum, even if the requirement of no rights violation is met and it is adopted a modified CA model in which “everyone finds everyone acceptable", there is another problem to be taken into account: the system should ban the possibility of “incorrect use of preferences".
An “incorrect use of preferences" is an expression of preferences which are in conflict with system compliance (COM): in this case higher-order policy goals or ethical principles.
This happens because countries can express indirect preferences that can discriminate, this for example happens by requiring language skills.
Compliance is also violated if agents can “game the system"~\cite{basshuysen}.


\subsection{Matching models as School Choice Problem}\label{matching-model-as-school-choice-problem}%

A SC model can be attained from the CA model by restricting the set of preference lists to the refugees, and defining priority lists for the countries.
Thus, an SC instance of a refugee-country matching problem is a 5-tuple \((R, C, q, P_r, Pri)\) where \(Pri = \{Pri(c_1), \ldots Pri(c_m)\}\) is the set of countries’ priority lists containing the same data as \(P_c\).
The rest of the definition is the same as the general model and the CA model.

Priority rankings are usually generated through a point system: if two applicants have identical points, the priority ranking may be determined through a lottery or continuous factors.
In the context  of the refugee match, it seems plausible to give a refugee who has been waiting longer for transfer  more points in all countries’ rankings than to a refugee with less waiting time spent, other things being  equal;
so waiting time since registration in the system could be used as a continuous variable to break non-strict priorities~\cite{basshuysen}.

The SC model respects the requirements of Stability and maximum cardinality and COM in that it does not allow the inclusion of requirements contrary to the higher-order policy goals or ethical principles~\cite{basshuysen}.


\subsection{Machine learning-based Matching}\label{machine-learning-based-matching}%

Let \(\pi_{ij}\) be the probability that the family \(i\) integrates properly in the country \(j\) and it can be estimated from \(P_c\) using a machine learning process.
We call the expected number of successfully integrated families, i.e., \(z (\mu) = \sum_{i \in F} \pi_{i \mu (i)}\), the objective function of \(\mu\). A feasible matching \(\mu^*\) is optimal if \(\mu^* \in \arg\max_{\mu} \{z (\mu)\}\).
Variable \(x_{ij}\) indicates whether the family \(i\) will be assigned to country \(j\).
Our goal is to find a feasible matching \(\mu\) that maximizes family welfare in terms of reported preferences \(P_r\) and at the same time ensures that \(z (\mu)\) is within a factor of \(\alpha\) of \(z \left( \mu^* \right) \) for a previously chosen \(\alpha \in [0,1]\).

A CRSD instance \(\hat {I} = (F, L, q, \pi, P, \gamma)\). The instance return \(\mu : F \rightarrow C \cup \{\emptyset\}\).
The general idea of CRSD is to let a family choose their match only from the set of remaining countries if it can be guaranteed an \(\alpha\)-approximation of the objective.
The constrained random serial dictatorship (CRSD) mechanism~\citet{bansak_2018} is a constrained version of the well known random serial dictatorship mechanism.
Algorithm is illustrated the~\ref{alg:crsd}.

Constrained rank value (CRV) mechanism it is a variant of the CRSD one that uses a rank function.
It is based on \(\hat {I} = ( F , C , P_r , \pi  ,v, \alpha )\) where \(v\) is a rank function express ranks assigning a values between 0 and 1 to positions in preference orders that is monotonically decreasing.
Algorithm is illustrated the~\ref{alg:crv}~\citet{olbergml}.

\begin{algorithm}
    \caption{Constrained Random Serial Dictatorship (CRSD)}\label{alg:crsd}
    \KwData{\( F , C , P_r , \pi  , \alpha \)}
    \KwResult{\(\mu\)}
    \SetKw{Break}{break}
    \( \mu ( i ) \leftarrow \emptyset \) for all \( i \in F \).

    \(\tiny
        \begin{aligned}
            z ^ * \leftarrow \operatorname{maximize} & \sum_{i \in F} \sum_{j \in C} \pi_{i j} a_{i j} \\[-1mm]
            \operatorname {subject}\operatorname{to} & \sum_{i \in F} a_{i j}=q_{j} & \forall j \in C \\[-1mm]
            & \sum_{j \in C} a_{i j}=1 & a_{i \mu(i)}=1 \\[-1mm]
            & a_{i\mu (i)} = 1 &  \forall i \in F(\mu) \\[-1mm]
            & a_{i j} \in\{0,1\} & \forall i \in F, \forall j \in C
        \end{aligned}
    \)%

    \( Q \leftarrow F \).

    \While{\( Q \neq \emptyset \)}{
        Remove randomly chosen \( i \in Q \) from \( Q \).

        \If{\( \left| F _ { j } ( \mu ) \right| < q _ { j } \)}{
            \( \mu ^ { \prime } \leftarrow \mu ; \mu ^ { \prime } ( i ) \leftarrow j \)

            \(\tiny
                \begin{aligned}
                    z ^ { \prime } \leftarrow \operatorname{maximize} & \sum_{i \in F} \sum_{j \in C} \pi_{i j} a_{i j} \\[-1mm]
                    \operatorname {subject}\operatorname{to} & \sum_{i \in F} a_{i j}=q_{j} & \forall j \in C \\[-1mm]
                    & \sum_{j \in C} a_{i j}=1 & a_{i \mu(i)}=1 \\[-1mm]
                    & a_{i\mu (i)} = 1 &  \forall i \in F(\mu) \\[-1mm]
                    & a_{i j} \in\{0,1\} & \forall i \in F, \forall j \in C
                \end{aligned}
            \)%

            \If{\( z ^ { \prime } \geq \alpha z ^ { * } \)}{
                \( \mu ( i ) \leftarrow j \)

                \Break
            }
        }
    }
\end{algorithm}

\begin{algorithm}
    \caption{Constrained Rank Value Mechanism (CRV)}\label{alg:crv}
    \KwData{\( F , C , P_r , \pi  ,v, \alpha \)}
    \KwResult{\(\mu\)}
    \(\tiny
        \begin{aligned}
            z \left( \mu ^ { * } \right) \leftarrow \operatorname{maximize} & \sum_{i \in F} \sum_{j \in C} \pi_{i j} a_{i j} \\[-1mm]
            \operatorname {subject}\operatorname{to} & \sum_{i \in F} a_{i j}=q_{j} & \forall j \in C \\[-1mm]
            & \sum_{j \in C} a_{i j}=1 & a_{i \mu(i)}=1 \\[-1mm]
            & a_{i\mu (i)} = 1 &  \forall i \in F(\mu) \\[-1mm]
            & a_{i j} \in\{0,1\} & \forall i \in F, \forall j \in C
        \end{aligned}
    \)%

    \(\tiny
        \begin{aligned}
            \mu \leftarrow \operatorname{maximize} & \sum_{i \in F} \sum_{j \in C} v\left(P_r(r_i)_j\right) a_{i j} \\[-1mm]
            \operatorname {subject}\operatorname{to} & \sum_{i \in F} a_{i j}=q_{j} & \forall j \in C \\[-1mm]
            & \sum_{j \in L} a_{i j}=1 & \forall i \in F \\[-1mm]
            & \sum_{i \in F} \sum_{j \in C} \pi_{i j} a_{i j} \geq \alpha z \left( \mu ^ { * } \right) \\[-1mm]
            & a_{i j} \in\{0,1\} & \forall i \in F, \forall j \in C
        \end{aligned}
    \)%

\end{algorithm}

